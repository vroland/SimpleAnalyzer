Das Simple\-Analyzer-\/\-Softwarepaket enthält Programme zur Auswertung physikalischer Versuche für debianbasierte Betriebssysteme. Mithilfe der enthaltenen Software sind Sie im Stande, Temperaturmessdaten aus einer .csv-\/\-Datei oder Messwerte des O\-Di\-S\-I-\/\-Instruments von Luna in ein einheitliches Format umzuwandeln und zusammenzuführen.

Über eine grafische Oberfläche ist es möglich, mithilfe der so aufbereiteten Daten Auswertungen wie eine Temperaturverteilung über ein dreidimensionales Modell oder das Bestimmen des Wärmegehalts vorzunehmen und den Versuch zu visualisieren.

Zur weiteren Nutzung der Ergebnisse können diese, beispielsweise als V\-T\-K-\/\-Datei oder P\-N\-G-\/\-Grafik, exportiert werden.

Quelltext, Handbuch, Dokumentation und Beispiele sowie Binärdateien finden Sie unter \href{https://github.com/vroland/SimpleAnalyzer}{\tt https\-://github.\-com/vroland/\-Simple\-Analyzer}.

\section*{Handbuch }

Im Handbuch zum Programm finden Sie Informationen zur Installation und Bedienung der Programme. Es liegt im pdf-\/\-Format unter simpleanalyzer-\/gui/\-Debug/simpleanalyzer-\/man.\-pdf vor und kann über das Hilfemenü in Simple\-Analyzer-\/\-G\-U\-I aufgerufen werden.

\section*{Dokumentation }

Eine Dokumentation für die Weiterentwicklung der Software befindet sich im Verzeichnis doc/html. 