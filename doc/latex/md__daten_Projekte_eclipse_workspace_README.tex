Das Simple\-Analyzer-\/\-Softwarepaket enthält Programme zur Auswertung physikalischer Versuche für debianbasierte Betriebssysteme. Mithilfe der enthaltenen können Sie Temperaturmessdaten aus einer .csv-\/\-Datei oder Messwerte des O\-Di\-S\-I-\/\-Instruments von Luna zusammenführen und weiter zu verarbeiten. Über grafische Oberfläche wird das Berechnen einer Temperaturverteilung eines dreidimensionalen Modells und das Bestimmen von physikalischen Größen und die Visualisierung des Versuchs ermöglicht. Zur weiteren Nutzung der Ergebnisse ist es möglich diese, beispielsweise als V\-T\-K-\/\-Datei oder P\-N\-G-\/\-Grafik, zu exportieren.

Quelltext, Handbuch, Dokumentation und Beispiele sowie Binärdateien finden Sie unter \href{https://github.com/vroland/SimpleAnalyzer}{\tt https\-://github.\-com/vroland/\-Simple\-Analyzer}.

\section*{Handbuch }

Im Handbuch zum Programm finden Sie Informationen zur Installation und Bedienung der Programme. Es liegt im pdf-\/\-Format unter simpleanalyzer-\/gui/\-Debug/simpleanalyzer-\/man.\-pdf vor und kann über das Hilfemenü in Simple\-Analyzer-\/\-G\-U\-I aufgerufen werden.

\section*{Dokumentation }

Eine Dokumentation für die Weiterentwicklung der Software befindet sich im Verzeichnis doc/html. 